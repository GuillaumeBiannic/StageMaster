\begin{thebibliography}{10}

\bibitem{ri11}
  	Riva, G., (2011). The Key to Unlocking the Virtual Body: Virtual Reality in the Treatment of Obesity and Eating Disorders. Journal of Diabetes Science and Technology.
  	Volume 5, Issue 2.
  			
\bibitem{ri13}
	Ferrer-García, M., Gutiérrez-Maldonado, J., Riva, G. (2013). Virtual reality based treatments in eating disorders and obesity: A review.  Journal of Contemporary Psychotherapy, 43, 207-221.
	
\bibitem{gu10}
 	Guardia, D., Lafarguea, G., Thomas, P., Dodin, V., Cottencin, O., \& Luyat, M. (2010).
 	Anticipation of body-scaled action is modified in anorexia nervosa.
 	Neuropsychologia, Volume 48, Issue 13, Pages 3961-3966
 \bibitem{lu14}
 	Luyat, M. (2014). Les apports de la psychologie cognitive et de la
 	neuropsychologie dans la compréhension de l'anorexie mentale. Journal de
 	thérapie comportementale et cognitive, 24, 114-121.
 	
 \bibitem{Bo98}
 	Botvinick, M., \& Cohen, J. (1998).
 	Rubber hands ‘feel’
 	touch that eyes see. Nature, 391, 756.
 	
  	
 \bibitem{bl10}
 	Lopez, C., \& Blanke, O. (2010). Quand l'esprit met le corps à distance. La
 	Recherche, 439, 48-51.
 	
 \bibitem{eh07}
 	Ehrsson H. H. (2007). The experimental induction of out-of-body experiences.
 	Science , 317-1048.
 
 \bibitem{le07}
 	Lenggenhager, B., Tadi, T. Metzinger, T., \& Blanke, O. (2007). Video ergo sum: manipulating
 	bodily self-consciousness. Science, 317, 1096–1099
 	
 
 \bibitem{sl09}
  	Mel Slater, M., Daniel Perez-Marcos, D., H. Henrik Ehrsson, H. H., \&
  	Maria V. Sanchez-Vives, M. V. (2009).Inducing illusory ownership of a virtual body.
  	 Frontiers in Neuroscience, 3(2), 214–220.
  	 
 \bibitem{sl08}
  Slater, M., Spanlang, B., Frisoli, A.,
  \& Sanchez-Vives, M.V. (2008).
  Virtual hand illusion induced by
  visual- proprioceptive and motor
  correlations. PLoS ONE 5(4): e10381.
   	
 \bibitem{pr14}
  	Preston, C., \& Ehrsson, H. H. (2014). Illusory changes in body size modulate body
  	satisfaction in a way that is related to non-clinical eating disorder
  	psychopathology. PLoS ONE 9(1): e85773.
  	
 \bibitem{ca02}
 	Cash, T. F., Fleming, E. C., Alindogan, J., Steadman L., \& Whitehead, A. (2002) Beyond
 	Body Image as a Trait: The Development and Validation of the Body Image
 	States Scale. Eat Disord: J Treat Prevent 10: 103–113.
 	
 \bibitem{zh09}
 	Zhong, Y. Q., Liu, H. Y., Jiang, J. F., \& Liu L. (2009). 3D Human Body Morphing Based on Shape Interpolation.
 	The 1st International Conference on Information Science and Engineering, p1027-1030. 
 	
 \bibitem{le01}
 	Lee, W., Magnenat-Thalmann N. (2001). Virtual Body Morphing
 	Computer Animation, The Fourteenth Conference on Computer Animation. Proceedings, p158-166 
 \bibitem{kn07}	
 	Knossow, D. (2007).Paramétrage et Capture Multicaméras du Mouvement Humain. 
 	Human-Computer Interaction. Institut National Polytechnique de Grenoble - INPG.
 	
 \bibitem{zo12}
 	Zong, C. (2012). Système embarquée de capture et analyse du mouvement humain durant la marche.
  	Automatic. Université Pierre et Marie Curie - Paris VI.
  	
  \bibitem{ze12}	
  	Zhang, Z. (2012). Microsoft Kinect
  	Sensor and Its Effect. MultiMedia, IEEE, vol. 19, no. 2, pp. 4-10
 	
 \bibitem{ch12}	
 	Chang, C., Lange, B., Zhang, M., Koenig, S., Requejo, P., Somboon, N.,
 	Sawchuk, A. A., \& Rizzo A. A. (2012). Towards Pervasive Physical Rehabilitation Using
 	Microsoft Kinect. In International Conference on Pervasive Computing Technologies for Healthcare (PervasiveHealth), p159-162.
 	
  \bibitem{li12}
  	Livingston, M. A., Sebastian, J., Ai, Z., \& Decker, J.  W. (2012). Performance Measurements for the Microsoft Kinect Skeleton. Proc. IEEE Virtual Reality Workshops,  pp.119 -120.
  	
\end{thebibliography}

