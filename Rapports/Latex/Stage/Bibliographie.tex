\section{\'{E}tat de l'art}

\subsection{Contexte}

Dans cette partie, nous allons voir ce qu'est l'anorexie mentale et ce qui a été réalisé dans le domaine de la sortie de corps en réalité virtuelle. Pour finir, nous verrons les contraintes à respecter pour créer une illusion en sortie de corps en réalité virtuelle pour aider les personne atteintes d'anorexie mentale.
\subsubsection{Anorexie mentale}

L'anorexie mentale est une maladie faisant partie des troubles du comportement alimentaire. Il s'agit une maladie psychiatrique grave qui touche principalement les femmes au moment de l'adolescence suite à une perte de poids et se présente essentiellement sous la forme de trois symptômes :
\begin{itemize}
\item Une restriction alimentaire entraînant une réduction drastique des apports énergétiques malgré les besoins physiologiques.
\item Une peur importante de reprendre du poids.
\item Une mauvaise représentation de son corps.
\end{itemize}

Les personnes touchées par cette maladie surestiment leur poids et leur silhouette, ce qui stoppe ou ralentit le processus de renutrition. Pour déterminer la source de cette mauvaise représentation du corps, Guardia et al. \cite{gu10} ont mis en place une expérience dans laquelle une personne est face à un mur sur lequel est projeté une porte dont la largeur varie. La personne doit alors dire si elle peut passer la porte sans tourner les épaules. Cette expérience a été faite sur des personnes en bonne santé et des personnes atteintes d'anorexie mentale. Ils ont pu observer que les personnes souffrant d'anorexie mentale se tournaient bien avant les autres personnes. Ces résultats tendent à indiquer que le schéma corporel, qui est une représentation sensorimotrice du corps sollicité pour gérer la posture et les actions du corps, serait faussé chez les patients.\\

 Comme le schéma corporel est supposé être basé sur une intégration multisensorielle, l'illusion de la main en caoutchouc ("\emph{Rubber Hand Illusion}") \cite{Bo98} (Voir Figure \ref{fig1}), qui se base sur la création de conflit dans le processus d'intégration multisensorielle, a été réalisée sur des patients souffrants d'anorexie mentale. L'illusion a pour but de faire croire à la personne que la main en caoutchouc fait partie de son corps. Pour cela, une des mains de la personne est cachée et une main en caoutchouc est placée devant elle. Ensuite, la main cachée et la main en caoutchouc sont touchées, par un bâton ou un pinceau par exemple, simultanément ce qui provoque un conflit entre ce qui est vu, la main en caoutchouc touchée par un pinceau, et ce qui est ressenti, la sensation de toucher créée par le pinceau sur la main cachée. Une fois l'illusion créée, on peut ne caresser que la main en caoutchouc. Cette illusion s'est avérée très forte chez les personnes atteintes d'anorexie mentale.\\

Pour aider les patients souffrants d'anorexie mentale, il faut d'abord les aider à corriger la mauvaise représentation qu'ils ont de leurs corps. Pour cela il faut pouvoir modifier la perception qu'ils ont de leur corps. L'illusion de la main en caoutchouc permet de le faire uniquement sur une partie du corps mais elle est très efficace sur eux. Donc la sortie de corps qui repose sur le même principe que cette illusion mais en l'étendant à tout le corps devrait également être efficace sur les personnes atteintes d'anorexie mentale.
\begin{figure}[h]
   	\centerline{\includegraphics[scale=0.4]{images/biblio/rubberhand}}
   	\caption{\label{fig1} Illusion de la main en caoutchouc \cite{lu14}}
\end{figure}


\subsubsection{Sortie de corps}

La sortie de corps est un phénomène dans lequel une personne a l'impression d'apercevoir le monde d'une position surélevée et de voir son corps à l'extérieur de ses limites physiques, d'avoir la sensation d'être désincarnée \cite{bl10}. Dans cette partie nous allons voir différents travaux réalisés pour reproduire ce phénomène en réalité virtuelle. En réalité virtuelle, l'illusion de sortie de corps cherche à créer une sensation d'appartenance d'un corps ou une partie de corps virtuel contrairement à la vrai sortie de corps qui se définit par une désincarnation.

\paragraph{Sortie de corps en réalité virtuelle}

Olaf Blanke et al. \cite{le07}, se sont appuyés sur l'illusion de la main en caoutchouc pour mettre au point leur expérience (voir Figure \ref{fig2}). Dans cette expérience, le sujet porte un casque de réalité virtuelle et est filmé de dos. Le film est envoyé en temps réel au casque porté par la personne ce qui fait qu'elle voit donc une image de son corps projeté devant elle. Le sujet est alors touché à répétition dans le dos par un bâton. Il voit alors une image virtuelle dans laquelle le dos de son corps est touché par un bâton et en même temps ressent une sensation de toucher dans le dos. L'expérience est réalisée dans deux conditions, une où l'image est synchronisée avec la stimulation tactile et une autre où un délai est ajouté pour créer un décalage entre le moment où la personne est touchée par le bâton et le moment où elle se voit touchée par le bâton dans son casque de réalité virtuelle. Avec la synchronisation, les sujets avaient l'impression que la sensation de toucher venait du fait que le corps virtuel qu’ils percevaient via le casque était touché par un bâton. \'{E}galement, en demandant aux sujets, après qu'ils se soient déplacés, de se remettre à la position où ils étaient, ils avaient tendance à se placer plus près de là où leur corps virtuel était projeté qu'il ne l'était en vrai. Ceci tend à montrer qu'ils se sont identifiés au corps virtuel. Par contre, dans le cas où il n'y avait pas de synchronisation, ces erreurs d'assimilation étaient plus rares.\\

Simuler une attaque sur le faux corps, avec un couteau ou un marteau, peut même augmenter la conductance de la peau ce qui montre une réponse émotionnelle lorsque le corps virtuel est menacé \cite{eh07}. Le même résultat a pu être obtenu en filmant un mannequin de dos touché par un bâton au lieu de filmer directement la personne de dos, même si dans ce cas-là il faut aussi que le sujet soit également touché dans le dos de manière synchronisée par rapport au mannequin \cite{le07}.\\

\begin{figure}[h]
   	\centerline{\includegraphics[scale=0.6]{images/biblio/oobRV}}
   	\caption{\label{fig2} Exemple de sortie de corps en Réalité Virtuelle \cite{bl10}}
\end{figure}

L'illusion de la main en caoutchouc a également était réalisée en réalité virtuelle \cite{sl09}, et elle a été réalisée sans utiliser de stimulation tactile mais en bougeant la main virtuelle pour qu'elle reproduise les mouvements de la main de la personne \cite{sl08}(Voir Figure \ref{fig3}). En utilisant un gant de données pour reproduire les mouvements de la main et des doigts de la personne, ils ont pu constater une tendance à identifier la main virtuelle comme étant une partie du corps. La synchronisation entre les mouvements réalisés par le sujet et ceux fait par la main virtuelle est importante pour créer ce sentiment d'appartenance. Cela permet de penser qu'une sortie de corps pourrait être réalisée en reproduisant le mouvement de la personne sur un avatar virtuel.


\begin{figure}[!h]
   	\centerline{\includegraphics[scale=0.4]{images/biblio/rhiRV}}
   	\caption{\label{fig3} Illusion de la main en caoutchouc en Réalité Virtuelle \cite{sl08}}
\end{figure}


\paragraph{Modification de la perception du corps}

En utilisant la sortie de corps avec des mannequins, Catherine Preston et H. Henrik Ehrsson ont réussi à modifier la satisfaction du corps chez des personnes non atteintes de trouble du comportement alimentaire \cite{pr14}. Pour cela une sortie de corps était réalisée sur les sujets en utilisant un mannequin "maigre" représentant 85\% de la masse corporel du sujet et un mannequin large représentant 115\% de la masse corporel. En utilisant le questionnaire \emph{Body Image States Scale} (BISS) \cite{ca02} qui permet de connaître la satisfaction qu'une personne éprouve par rapport à son corps à l'instant présent, ils ont pu constater une modification dans la satisfaction du corps chez les sujets. Avec la sortie de corps, il a donc été possible chez ces personnes de modifier la satisfaction de leur corps et donc de modifier la perception qu'ils avaient de leur corps. Cependant, l'utilisation de mannequins n'est pas pratique car on ne peut pas changer leurs apparences aisément au niveau de la silhouette. 

\subsubsection{Bilan}
Comme on a pu le voir précédemment, on peut modifier la perception du corps avec la sortie de corps et c'est justement ce que nous souhaitons faire pour aider les personnes atteintes d'anorexie mentale. Seulement l'utilisation de mannequin est limitant, alors qu'avec un modèle 3D d'un corps humain il serait notamment possible de modifier sa silhouette en temps réel. Pour réaliser cela, il faudra également pouvoir capturer les mouvements de la personne et les reproduire avec le modèle 3D. En regardant ce qui a été fait sur la sortie de corps en réalité virtuelle, des sollicitations tactiles sont nécessaires pour créer le phénomène, ce qui signifie qu'il faudra également capturer les mouvements de l'objet (un bâton par exemple) pour réaliser ces actions et reproduire l'objet et ses mouvements dans l'environnement 3D. Cependant, dans le cas de la main en caoutchouc en réalité virtuelle, le fait que la main virtuelle reproduise les mouvements de la vrai main de l'utilisateur permet de se passer des sollicitations tactiles ce qui pourrait aussi être le cas de la sortie de corps. Enfin le dernier point important est la synchronisation, il ne faut pas qu'il y ait de décalage entre ce qui est vu et, soit la sensation de toucher, soit le mouvement effectué, pour créer le phénomène de sortie de corps.

\subsection{Avatar et environnement 3D}
Dans cette section, nous allons voir l'effet que peut avoir un avatar en fonction de son réalisme ainsi que l'impact que peut avoir un environnement virtuel sur notre perception. 
\subsubsection{Apparence de l'avatar}
En utilisant un avatar virtuel, on peut voir apparaitre un effet appelé la \emph{Uncanny Valley} \cite{mo12}. Cet effet représente initialement le fait que dans le domaine de la robotique, plus un robot ressemble à un humain, plus ces défauts gêneront l'utilisateur qui interagit avec lui. Cet effet existe aussi avec les avatar virtuel \cite{mc12} et est accentué lorsque l'avatar réalise des mouvements. Dans cette étude, on peut constater qu'un avatar très réaliste semble familier à l'utilisateur mais qu'il s'agit de l'avatar moyennement réaliste qui semble le moins familier. L'étude regarde aussi l'effet que peut avoir l'apparition d'artefact dans l'animation de l'avatar. Les problèmes d'animations ont moins d'impact lorsque l'avatar n'est pas réaliste et il s'agit de l'avatar moyennement réaliste qui est le plus déplaisant en cas de mauvaise animations. L'interaction avec un avatar ayant une apparence humaine mais pas suffisamment réaliste peut repousser l'utilisateur.
\subsubsection{Perception dans un environnement virtuel}
Pour voir l'impact de la qualité graphique d'un environnement 3D sur la présence ressenti par l'utilisateur, Slater et al. \cite{sla09} on mit des sujets devant un gouffre virtuel, et on analysé la sensation de présence dans l'environnement 3D via questionnaire et les valeurs de conductances de la peau. Suivant les sujets, l'environnement avait des ombres et des réflections dynamiques ou aucunes des deux. Les résultats montre que la présence des ombres et des réflections augmentent ce sentiment de présence.
\subsubsection{Bilan}
Le réalisme de l'apparence de l'avatar peut avoir un impact sur la capacité de l'utilisateur de se lier à lui. Cependant dans le cas d'une sortie de corps où l'utilisateur voit son corps virtuel de dos, rend le visage de l'avatar non visible et ainsi il ne voit pas que l'expression faciale ne corresponde pas à la sienne. Ceci devrait réduire le risque de produire l'effet \emph{Uncanny Valley}. Des critères graphiques comme la présence d'ombre dynamique pour l'avatar pourraient augmenter la présence de l'utilisateur et aider à la création de l'effet de sortie de corps.

\subsection{Modification d'un corps virtuel}
Dans cette partie nous allons voir différentes méthodes pour modifier l'apparence d'un corps virtuel.

\subsubsection{Shape Interpolation}

La \emph{shape interpolation} permet de passer d'un modèle de corps humain à un autre en transformant les deux modèles en des ensembles de points et en calculant entre chaque point correspondant des deux modèles des points intermédiaires \cite{zh09}. Pour pouvoir réaliser cette transformation, il faut d'abord ré-échantillonner les deux modèles pour les transformer en "nuages" de points. Pour cela, un découpage du modèle est effectué en suivant l'axe du squelette du modèle. Par exemple, pour l'avant-bras une série de coupe parallèle va être réalisée en suivant la partie du squelette connectant le coude et le poignet. Le nombre de coupe réalisé sur un membre doit être identique sur les deux modèles. Une fois ceci réalisé, chaque partie du modèle est représentée par un ensemble de tranche. Ensuite le ré-échantillonnage se fait en trois étapes :
\begin{itemize}
\item \'{E}tape 1 : Pour chaque tranche, on calcule un cercle englobant cette partie du corps.
\item \'{E}tape 2 : On divise le cercle en \emph{n} sections, et on crée \emph{n} rayons qui partent du cercle et vont jusqu'au centre de la tranche.
\item \'{E}tape 3 : On crée un point à l'endroit où chaque rayon traverse la surface du modèle.
\end{itemize}

Une fois que les deux modèles ont été transformés en ensemble de points, Il faut alors calculer un ensemble de points intermédiaires via une interpolation linéaire. \`{A} partir de ces points, la surface du modèle intermédiaire peut être créée en effectuant une triangulation. Sur la figure \ref{fig4} on peut voir des modèles créés avec cette technique.
\begin{figure}[!h]
   	\centerline{\includegraphics[scale=0.4]{images/biblio/shapeInterpolation2}}
   	\caption{\label{fig4} Modèles de bases (entourés en bleu) et modèles créés par interpolation \cite{zh09}}
\end{figure}
\subsubsection{\emph{Morphing 3D}}
Pour réaliser un \emph{morphing} d'un humain virtuel à un autre en 3D, il faut réaliser plusieurs étapes \cite{le01} :
\begin{itemize}
\item \emph{Morphing} du squelette.
\item \emph{Morphing} de la forme.
\item \emph{Morphing} des coordonnées de la texture.
\item \emph{Morphing} de l'image de la texture.
\end{itemize}
Pour réaliser le \emph{morphing} du squelette et de la forme, il faut que les deux modèles utilisent la même structure pour définir le squelette et la forme. Ensuite les coordonnées du squelette et de la forme peuvent être calculées en utilisant une interpolation linéaire en 3D.
Pour réaliser le \emph{morphing} de la texture, il faut d'abord interpoler les coordonnées de la texture en utilisant une interpolation linéaire en 2D. Le morphing de l'image est réalisé en utilisant des techniques de triangulation. Sur la figure \ref{fig8} on peut voir des modèles créés avec cette technique.
\begin{figure}[!h]
   	\centerline{\includegraphics[scale=0.4]{images/biblio/morphing}}
   	\caption{\label{fig8} Modèles avec différentes formes et la même texture \cite{le01}}
\end{figure}
\subsubsection{Modélisation paramétrique}
\subsubsection{Bilan}
La \emph{shape interpolation} permet de passer d'un modèle de corps humain à un autre même si ils ne sont pas définis suivant la même structure. En effet, les points utilisés pour réaliser l'interpolation sont créés avec cette technique. Par contre, il n'y a pas de modification de la texture contrairement au \emph{morphing 3D} et le nombre de calcul est plus important que l'autre technique à cause du calcul des points. Comme les modèles utilisés seront connus, la structure pour définir le squelette et la forme sera identique pour les deux modèles et donc le \emph{morphing 3D} est plus approprié dans notre contexte.

\subsection{Capture de mouvement}
Dans cette partie, nous allons voir trois grandes classes de système de capture : système mécanique, système magnétique et système optique\cite{kn07}\cite{zo12}. Nous allons d'abord décrire les deux premiers systèmes. Ensuite, nous nous attarderons sur les systèmes optiques en regardant les systèmes optiques avec marqueurs et la \emph{Kinect} puis nous comparerons les performances des deux systèmes. L'étude se limite à ces deux dispositifs car sont ceux disponibles pendant mon stage.
\subsubsection{Système mécanique}
 Le système mécanique utilise un exosquelette (Voir Figure \ref{fig7}) qui doit être porté par la personne dont on capture le mouvement. Le mouvement est mesuré grâce à des potentiomètres qui sont placés au niveau des articulations ce qui simplifie le traitement de l'information. Son champ d'action est très grand car il n'est relié à aucun appareil externe et il n'y a pas de perturbation au niveau de la mesure. Par contre ce système est très intrusif car l'exosquelette encombre énormément la personne et l'empêche de faire des mouvements rapides.
 \begin{figure}[!h]
    	\centerline{\includegraphics[scale=0.8]{images/biblio/captureMecaTest}}
    	\caption{\label{fig7} Exemple d'exosquelette}
 \end{figure}
 %\newpage
\subsubsection{Système magnétique}
Pour ce système, des capteurs sensibles à un champ magnétique produit par un émetteur sont posés sur la personne (Voir Figure \ref{fig9}). La positon et l'orientation des capteurs sont mesurables et la vitesse d'acquisition est très rapide. Ce dispositif est assez peu encombrant par contre il est très sensible à des perturbations pouvant être créées par des objets métalliques ce qui crée des contraintes sur l'environnement où est utilisé ce système.
\begin{figure}[!h]
   	\centerline{\includegraphics[scale=0.4]{images/biblio/magne}}
   	\caption{\label{fig9} Combinaison d'un système magnétique}
\end{figure}

\paragraph{\emph{Razer Hydra}}
La \emph{Razer Hydra} \cite{ku12} est un appareil qui utilise un champ magnétique pour obtenir la position et la rotation de deux contrôleurs, un pour chaque main. Un capteur magnétique sert de base à l'appareil et les contrôleurs ont un rayon d'action d'environ deux mètres. l'appareil offre une précision à un millimètre près pour ce qui est de la position et de un degrés près pour ce qui est de la rotation. La \emph{Razer Hydra} permet donc de capturer les mouvements des mains de l'utilisateur mais via un procédé intrusif car il faut tenir les deux contrôleurs.
%\begin{figure}[!h]
%   	\centerline{\includegraphics[scale=0.2]{images/biblio/razerHydra}}
%   	\caption{\label{Razer} Capteur magnétique \emph{Razer Hydra}}
%\end{figure}

\subsubsection{Système optique}

\paragraph*{Système avec marqueurs}
Dans ce système, l'utilisateur porte une combinaison possédant des marqueurs réfléchissants (Voir Figure \ref{fig5}). Gr\^{a}ce à plusieurs caméras infrarouge, la position de chaque marqueur est récupéré en 3D. Pour cela, chaque caméra émet une lumière infrarouge qui va être réfléchie par le marqueur permettant ainsi d'obtenir une image 2D de celui-ci. En combinant les différentes images obtenues, la position 3D du marqueur peut être calculée. Ce système permet d'avoir une bonne précision et de capturer des mouvements rapides. Ce système est moins encombrant que les deux précédents.
\begin{figure}[!h]
   	\centerline{\includegraphics[scale=0.35]{images/biblio/captureoptic}}
   	\caption{\label{fig5} Dispositif de système optique avec marqueurs}
\end{figure}
\paragraph*{\emph{Kinect}}
La \emph{Microsoft Kinect} est un système de capture de mouvement ne nécessitant pas de marqueurs et embarquant notamment une camera RGB et un capteur de profondeur \cite{ze12}. Le capteur de profondeur permet, en utilisant des rayons infrarouges, d'obtenir l'image de la scène en prenant en compte la profondeur (Voir Figure \ref{fig6}). L'image obtenue est traitée et les formes sont reconnues ce qui permet d'appliquer un squelette sur les formes humaines et ainsi suivre les mouvements.
\begin{figure}[!h]
   	\centerline{\includegraphics[scale=1.0]{images/biblio/depthsensor}}
   	\caption{\label{fig6} Exemple d'image obtenu avec le capteur de profondeur}
\end{figure}
\paragraph*{Performances}
Dans l'article \cite{ch12}, les auteurs cherchent à comparer les performances obtenues avec un système optique avec marqueurs \emph{OptiTrack} et la \emph{Kinect} pour leur application d'aide à la réadaptation physique. Dans leur test, le sujet doit effectuer divers mouvements au niveau des mains, des coudes et des épaules qui sont capturés en même temps par le système \emph{OptiTrack} et le système \emph{Kinect}. Ils ont pu constater que la \emph{Kinect} capture très mal les mouvements effectués au niveau des épaules contrairement au système optique avec marqueurs. Cela est dû au fait que la \emph{Kinect} n'utilise qu'une seule caméra contrairement à l'autre système qui en utilise plusieurs (16 dans ce test). En comparant les trajectoires obtenues pour les coudes et mains, ils ont pu voir que les résultats des deux systèmes étaient proches. La latence relative entre les deux systèmes a également été calculée et le système \emph{OptiTrack} était plus rapide 50 ms. La \emph{Kinect} n'offre pas la possibilité de reconnaitre la position et le mouvement des doigts. La portée de la \emph{Kinect} est d'environ de 0.5 à 4 mètres et la portée optimale est d'environ 1.2 à 3.5 mètres. La distance à laquelle se tient l'utilisateur influence beaucoup la performance du \emph{Kinect} \cite{li12}.

\subsubsection{Bilan}
Les systèmes optiques avec marqueurs sont rapides et permettent une bonne précision ce qui est nécessaire pour notre projet. Cependant ce système implique le port d'une combinaison avec marqueurs et cela pourrait gêner les utilisateurs dans notre contexte. Comme on a pu le voir, la \emph{Kinect}, bien qu'ayant des performances inférieurs à ceux des systèmes optiques avec marqueurs, est suffisamment fiable et rapide. Par contre, la \emph{Kinect} peut ne pas capturer certains mouvements dû à sa seule caméra. Il faudra donc penser au placement de la caméra lors de la capture. Comme la \emph{Microsoft Kinect} est un dispositif plus simple à mettre en place, elle peut être utilisée dans plus d'environnement qu'un système optique avec marqueurs.



\subsection{Synthèse}

Comme nous l'avons vu, les patients atteints d'anorexie mentale doivent prendre conscience de leur maigreur pour que le processus de renutrition puisse aboutir. La sortie de corps, qui consiste à faire croire à une personne qu'un corps virtuel est son vrai corps, pourrait aider efficacement à modifier la perception qu'ils ont de leurs corps. Cette illusion est basée sur le même principe que celle de la main en caoutchouc et comme cette dernière s'est montrée efficace sur les personnes atteintes d'anorexie mentale, on peut en déduire que la sortie de corps est réalisable sur les patients et qu'elle pourrait même être particulièrement efficace.\\

Nous avons vu que réaliser la sortie de corps en réalité virtuelle était possible en créant un conflit entre ce que le sujet voit et ce qu'il ressent physiquement. Ainsi, en créant des stimulations tactiles sur le sujet en le touchant avec un bâton par exemple, pendant qu'il voit les mêmes stimulations tactiles réalisées sur le faux corps, que ce soit son propre corps filmé de dos ou celui d'un mannequin, on peut faire ressentir l'illusion de sortie de corps. Pour la main en caoutchouc, l'illusion est réalisable en réalité virtuelle sans utiliser de stimulation tactile. \`{A} la place, il reproduisait avec la main virtuelle les mouvements que l'utilisateur faisait avec sa vrai main et ils sont réussi à obtenir des résultats similaires. Reproduire les mouvements de l'utilisateur sur le corps virtuel en temps réel pourrait peut-être permettre de créer l'illusion de sortie de corps. \\

L'effet d'appartenance au corps virtuel créé par la sortie de corps est démontré par le fait qu'après l'expérience, le sujet se place plus près de là où il voyait le corps virtuel qu'il ne l'était réellement. De plus, il a été observé que simuler une attaquer sur le faux corps augmente la conductivité de la peau ce qui montre qu'une réponse émotionnelle a lieu lors que le corps virtuel est en "danger". Preston et Ehrsson \cite{pr14} ont également montré que l'utilisation de cette illusion pouvait changer la perception qu'une personne a de son corps et d'en modifier la satisfaction bien que l'utilisation de mannequin leur imposait des limites. Il existe des techniques pour modifier la silhouette d'un corps virtuel comme le \emph{morphing 3D} et réaliser cette modification pendant que le sujet ressent l'illusion de sortie de corps pourrait avoir un impact plus important sur sa perception.\\

Pour capturer le mouvement, \emph{Kinect} est un système simple à mettre en place, il n'y a qu'une seule caméra et il n'y a pas de contrainte sur l'environnement à part la distance à laquelle on se trouve du dispositif. Par contre, la présence d'une seule caméra fait que certains mouvements ne sont pas vus et le dispositif ne capte pas avec autant de précision les mouvements qu'un système optique avec marqueurs. Ce dernier quant à lui, permet d'obtenir des mouvements précis si la personne est équipée de suffisamment de marqueurs et la fréquence d'acquisition des données est élevé. Comme il nécessite plusieurs caméras, il y a moins de risque que certains mouvements ne soit pas vus. En revanche, il s'agit d'un système qui demande plus de matériel et qui est intrusif car la personne doit porter des marqueurs et généralement une combinaison.
